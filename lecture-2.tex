\documentclass[main]{subfiles}

\begin{document}

\section{Лекция 2}

\begin{theorem}
	Определения $(1)$--$(4)$ непрерывности отображения в точке равносильны.
\end{theorem}

\begin{proof}
	Для начала заметим, что всегда $ B \ssq \W $. Таким образом, уже можно формально установить
	четыре следования: $\ft{1}{2}$, $\ft{3}{4}$, $\ft{3}{1}$, $\ft{4}{2}$. Действительно, если при некотором условии
	найдется окрестность во множестве $ B_1 $, то и во множестве $ \W_1 $ при том же условии тем более найдется.
	Также, если для любой окрестности во множестве $ \W_2 $ что-то верно, то оно же верно и для любой окрестности во
	множестве $ B_2 $. Таким образом, достаточно доказать только $\ft{2}{3}$.
	\begin{multiproof}
		\item[$\ft{2}{3}$] Рассмотрим такое произвольное множество $ V \in \W_2 $, что существует такое $ V' \in B_2 $,
			что $ V' \ssq V $ и $ f(x) \in V' $. Тогда из (2) для множества $ V' $ найдется такое множество
			$ U \in \W_1 $, что $ f(U) \ssq V' $ и $ x \in U $, а значит, найдется и такое множество $ U' \in B_1 $,
			что $ U' \ssq U $ и $ x \in U' $. Но тогда $ f(U') \ssq f(U) \ssq V' \ssq V $, что и требовалось.
	\end{multiproof}
\end{proof}

\subsection{Связность топологических пространств}

\begin{definition}
	Топологическое пространство $ (T, \W) $ называется \emph{линейно связным}, если любые две его точки можно соединить
	непрерывным путем. То есть для любых точек $ x_1, x_2 \in T $ существует такое непрерывное отображение
	$ f \colon [0; 1] \to T $, что $ f(0) = x_1 $ и $ f(1) = x_2 $.
\end{definition}

\begin{definition}
	Топологическое пространство $ (T, \W) $ называется \emph{несвязным}, если существуют такие множества
	$ U_1, U_2 \in \W $, что $ U_1, U_2 \neq \Emp $, $ U_1 \cap U_2 = \Emp $ и $ U_1 \cup U_2 = T $.
\end{definition}

\begin{remark}
	Эти две части дополненяют друг друга, а значит, одновременно являются замкнутыми и открытыми.
\end{remark}

\begin{definition}
	Топологическое пространство $ (T, \W) $ называется \emph{связным}, если оно не является несвязным.
\end{definition}

\begin{theorem} \label{the.2.1}
	Путь $ (T, \W) $ --- топологическое пространство. Тогда если оно является линейно связным, то оно является связным.
\end{theorem}

\begin{lemma} \label{lem.2.1}
	Пространство $[0; 1]$ со стандартной топологией связно.
\end{lemma}

\begin{proof}[Доказательство теоремы \ref{the.2.1}]
	Предположим, что $ T $ несвязно. Тогда $ T = U_1 \sqcup U_2 $, где $ U_1 $, $ U_2 $ открыты, непусты, не
	пересекаются. Рассмотрим произвольные точки $ x_1 \in U_1 $ и $ x_2 \in U_2 $. Так как пространство линейно
	связно, то существует такое непрерывное отображение $ f \colon [0; 1] \to T $, что $ f(0) = x_1 $ и $ f(1) = x_2 $.
	По свойствам прообраза $ f^{-1}(U_1) $, $ f^{-1}(U_2) $ также непусты, неперескаются и в объединении дают
	$[ 0; 1] $. Ввиду непрерывности отображения $ f $ и открытости множеств $ U_1, U_2 $ они также и открыты.
	Это противоречит связности отрезка $ [0; 1] $. Таким образом, предположение неверно, а значит, $ T $ связно.
\end{proof}

\begin{proof}[Доказательство леммы \ref{lem.2.1}] Предположим, что оно несвязно, тогда его можно представить в виде
	$ [0; 1] = A \sqcup B $, где $ A $, $ B $ открыты, непусты, не пересекаются. Без ограничения общности предположим,
	что $ 1 \in B $. Положим $ c = \sup A $. Пусть $ c \in (0; 1) $. Предположим, что $ c \in A $.
	Тогда, так как $ A $ открыто, то $ c $ содержится в $ A $ вместе с некоторой окрестностью,
	то есть такое существует $ \Eps > 0 $, что $ c + \Eps \in A $. Это противоречит тому, что $ c = \sup A $.
	Значит, предположение неверно, и $ c \in B $. Тогда аналогичные рассуждения показывают, что
	$ (c - \Eps; c] \ssq B $, что противоречит тому, что $ c = \sup A $. Пусть теперь $ c = 1 $, тогда также ввиду
	открытости множества $ B $ имеем $ (1 - \Eps; 1] \ssq B $, чего не может быть. Если же $ c = 0 $, то тогда
	$ A = \Set{0} $ --- это множество не является открытым.	Таким образом, исходное предположение неверно,
	а значит, пространство $ [0; 1] $ связно.
\end{proof}

\ImageConnectivity

\begin{proof}
	Пусть $ (T_1, \W_1) $, $ (T_2, \W_2) $ --- топологические пространства, $ T_1 $ связно, а отображение
	$ f \colon T_1 \to T_2 $ --- непрерывно. Предположим, что $ f(T_1) $ несвязно, то есть
	$ f(T_1) = V_1 \sqcup V_2 $, где $ V_1 $, $ V_2 $ открыты, непусты, не пересекаются. Тогда
	$ T_1 = f^{-1}(V_1) \sqcup f^{-1}(V_2) $, где эти два множества также открыты, непусты, не пересекаются.
	Это противоречит связность пространства $ T_1 $.
\end{proof}

\begin{restatable}[из домашнего задания]{exercise}{ExcII}
	Пусть $ (T_1, \W_1) $, $ (T_2, \W_2) $ --- топологические пространства, а $ f \colon T_1 \to T_2 $ --- некоторое
	отображение. Определим $ g \colon T_1 \to f(T_1) $ следующим образом: для любого $ x \in T_1 $ положим
	$ g(x) = f(x) $. Тогда если отображение $ f $ непрерывно, то отображение $ g $ также непрерывно.
\end{restatable}

\begin{problem}
	Отрезок $ [0; 1] $ и интервал $ (0; 1) $ не гомеоморфны.
\end{problem}

\begin{solution}
	Предположим, что эти пространства гомеоморфны, то есть существует гомеоморфизм $ f \colon [0; 1] \to (0; 1) $.
	Положим $ c = f(0) \in (0; 1) $. Тогда $ f_{(0;1]} \colon (0; 1] \to (0; c) \cup (c; 1) $ --- тоже гомеоморфизм,
	однако в этом случае $ f $ непрерывно, $ (0; 1] $ связно, а $ (0; c) \cup (c; 1) $ не связно, чего не может быть.
	Таким образом, отрезок и интервал не гомеоморфны.
\end{solution}

\end{document}
