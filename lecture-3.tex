\documentclass[main]{subfiles}

\begin{document}

\resetcounters

\section{}

\subsection{Аксимоматическое определение топологических пространств}

\begin{definition}
	\emph{База топологии} (или \emph{система окрестностей}) на множестве $ T $ --- это семейство $ B $ подмножеств
	множества $ T $, удовлетворяющее следующим аксиомам:
	\begin{enumerate}
		\item полнота покрытия: для любой точки $ x \in T $ существует окрестность $ U \in B $, содержащая эту точку;
		\item достаточная измельченность: для любых двух окрестностей $ U_1, U_2 \in B $ существует такое множество
			$\Nice{A} \ssq B $, что $ U_1 \cap U_2 = \bigcap_{V \in \Nice{A}} V $.
	\end{enumerate}
\end{definition}

\begin{remark}
	Второе свойство более канонично записывается в следующем виде: для любых двух окрестностей $ U_1, U_2 \in B $ и
	любой точки $ x \in U_1 \cap U_2 $ существует такая окрестность $ U \in B $, что $ x \in U $ и
	$ U \ssq U_1 \cap U_2 $. Эти две формулировки эквивалентны.
\end{remark}

\begin{restatable}{exercise}{ExcIII}
	Доказать эту эквивалентность.
\end{restatable}

\begin{definition}
	\emph{Топологией на множестве $ T $} называется семейство $ \W $ подмножеств множества $ T $,
	удовлетворяющая следующим аксиомам:
	\begin{enumerate}
		\item $ T \in \W $;
		\item для любого множества $ \Nice{A} \ssq \W $ верно $ \bigcup_{V \in \Nice{A}} V \in \W $;
		\item для любых множеств $ U_1, U_2 \in \W $ верно $ U_1 \cap U_2 \in \W $.
	\end{enumerate}
\end{definition}

\begin{restatable}{exercise}{ExcIV}
	Топология является базой топологии.
\end{restatable}

\begin{definition}
	Топологией на множестве $ T $, \emph{построенной по базе топологии} $ B $, назывется множество
	$ \W(B) = \Set{ \bigcup_{V \in \Nice{A}} \mid \Nice{A} \ssq B } $.
\end{definition}

\begin{restatable}{exercise}{ExcV}
	Проверьте, что это множество удовлетворяет аксиомам топологии.
\end{restatable}

\begin{restatable}{exercise}{ExcVI}
	Если $ (T, \W) $ --- топологическое пространство, то $ \W = \W(\W) $.
\end{restatable}

Полезно уметь получать топологию на подмножестве топологического пространства по заданной на нем топологии.
Один из возможных вариантов следующий: пусть $ (T, \W) $ --- топологическое пространство, $ B $ --- система
окрестностей $ \W = \W(B) $, $ T' \ssq T $. Тогда <<топологией>> на $ T' $ назовем топологию по <<базе>>
$ B' = \Set{ V \in B \mid V' \ssq T' } $. Это определение некорректно, то есть $ B' $ может не быть системой
окрестностей.

\begin{example}
	Например, пусть $ T = \R^2 $ и $ T' = \R $, на $ T $ топология стандартная. Однако в $ \R $ нет
	множеств, открытых в $ \R^2 $, так как любое подмножество множества $ \R $ замкнуто в $ \R^2 $.
\end{example}

\begin{definition}
	Пусть $ (T, \W) $ --- топологическое пространство, $ B $ --- база топологии $ \W $, $ T' \ssq T $.
	Топологией, \emph{индуцированной на $ T' $ топологией $ \W $}, называется топология, построенная по базе
	$ B' = \Set{ V \cap T' \mid V \in B } $.
\end{definition}

\begin{restatable}{exercise}{ExcVII}
	Докажите корректность этого определения. То есть покажите, что, во-первых, $ B' $ является базой топологии, а
	во-вторых, что получаемая топология не зависит от выбранной базы $ B $.
\end{restatable}

Рассмотрим теперь декартово произведение топологических пространств. Надо научиться индуцировать на него топологию.

\begin{definition}
	Пусть $ (T_1, \W_1) $, $ (T_2, \W_2) $ --- топологические пространства, тогда топологией \emph{на декартовом
	произведении} $ T_1 \times T_2 $ топологических пространств называется топология, построенная по базе топологии
	$ B = \Set{ B_1 \times B_2 \mid B_1 \in \W_1, B_2 \in W_2 } $.
\end{definition}

\begin{restatable}{exercise}{ExcVIII}
	Проверьте, что $ B $ --- база топологии.
\end{restatable}

\begin{theorem} \label{the.3.1}
	Пусть $ (T, \W) $, $ (T_1, \W_1) $, $ (T_2,\W_2) $ --- топологические пространства, заданы отображения
	$ f_1 \colon T \to T_1 $, $ f_2 \colon T \to T_2 $ и $ f \colon T \to T_1 \times T_2 $,
	причем $ f(x) = (f_1(x), f_2(x)) $ для любого $ x \in T $. Тогда отображение $ f $ непрерывно
	$\iaoi$ отображения $ f_1 $ и $ f_2 $ непрерывны.
\end{theorem}

\begin{lemma} \label{lem.3.1}
	Пусть $ (T_1, \W_1) $, $ (T_2, \W_2) $ --- топологические пространства, $ \W_2 = \W(B_2) $,
	$ f \colon T_1 \to T_2 $. Тогда если для любого множества $ U \in B_2 $ множество $ f^{-1}(U) $ открыто,
	то отображение $ f $ непрерывно.
\end{lemma}

\begin{proof}[Доказательство теоремы \ref{the.3.1}] \leavevmode
	\begin{multiproof}
		\item[$\Then$] Ясно, что для любого $ U \ssq T_1 $ верно $ f^{-1}_1(U) = f^{-1}( U \times T_2 ) $. Пусть
			теперь множество $ U $ открыто в $ T_1 $. Так как $ T_2 $ открыто в $ T_2 $, то по определению
			$ U \times T_2 \in B $ открыто в $ T_1 \times T_2 $. Значит, в силу непрерывности отображения $ f $,
			множество $ f^{-1}_1(U) = f^{-1}( U \times T_2 ) $ открыто, что и требовалось для непрерывности
			отображения $ f_1 $. Аналогично отображение $ f_2 $ также непрерывно.
		\item[$\If$] Рассмотрим произвольное множество $ U \in B $. Тогда $ U = U_1 \times U_2 $, где $ U_1 \in \W_1 $,
			$ U_2 \in \W_2 $. Но тогда $ f^{-1}(U) = f^{-1}_1(U_1) \cap f^{-1}_2(U_2) $ --- открыто как пересечение
			открытых множеств (так как $ f_1, f_2 $ непрерывны). По лемме этого достаточно для непрерывности
			отображения $ f $.
	\end{multiproof}
\end{proof}

\begin{proof}[Доказательство леммы \ref{lem.3.1}]
	Каждая точка лежит в некоторой окрестности, а значит, для любой точки $ f(x) \in T_2 $ существует такая
	окрестность $ U \in B_2 $, что $ f(x) \in U $. Тогда по условию $ f^{-1}(U) $ открыто и содержит точку $ x $.
	Более того, из теории множество известно $ f(f^{-1}(U)) \ssq U $, а значит, $ f $ непрерывно в точке $ x $
	по определению. Таким образом, функция непрерывна в каждой точке, тогда по доказанной ранее теореме она
	непрерывна.
\end{proof}

\begin{definition}
	Пусть $ T \neq \Emp $ --- некоторое множество, $ \mu \colon T^2 \to \R $ --- метрика на нем. Тогда
	топологией на множестве $ T $, \emph{индуцированной метрикой} $ \mu $, называется топология по базe
	$ B = \Set{ B_\Eps(x) \mid x \in T; \Eps > 0 } $, где $ B_\Eps(x) = \Set{ x' \in T \mid \mu(x, x') < \Eps } $.
\end{definition}

\begin{theorem}
	Множество $ B $ является базой топологии.
\end{theorem}

\begin{proof} Требуется доказать две аксиомы из определения базы топологии.
	\begin{enumerate}
		\item Для любой точки $ x \in T $ существует окрестность $ B_1(x) $, содержащая точку $ x $,
			так как по определению метрики $ \mu(x, x) = 0 < 1 $.
		\item Воспользуемся второй формулировкой аксиомы. Пусть $ x_1, x_2 \in T $. Рассмотрим две произвольные
			окрестности $ U_1 = B_{\Eps_1}(x_1), U_2 = B_{\Eps_2}(x_2) $ и точку $ x_3 \in U_1 \cap U_2 $. Положим
			$ \Eps_3 = \min \Set{ \Eps_1 - \mu(x_1, x_3), \Eps_2 - \mu(x_2, x_3) } > 0 $ и
			$ U_3 = B_{\Eps_3}(x_3) \in B $. Осталось доказать, что $ U_3 \ssq U_1 \cap U_2 $. Действительно, пусть
			$ x \in U_3 $, тогда
			\[ \mu(x, x_1) \Le \mu(x, x_3) + \mu(x_3, x_1) < \Eps_3 + \mu(x_3, x_1) \Le
				\Eps_1 - \mu(x_3, x_1) + \mu(x_3, x_1 ) = \Eps_1. \]
			Аналогично $ \mu(x, x_2) < \Eps_2 $, что и требовалось.
	\end{enumerate}
\end{proof}

\begin{definition}
	Топологическое пространство $ (T, \W) $ называется \emph{дискретным} (а топология --- \emph{дискретной}),
	если $ \W = 2^T $.
\end{definition}

\begin{definition}
	Топологическое пространство $(T, \W)$ называется \emph{тривиальным} (а топология --- \emph{антидискретной}),
	если $ \W = \Set{ \Emp, T } $.
\end{definition}

Многие используемые топологические пространства являются метрическими (то есть топология на них индуцирована некоторой
меирикой). Рассмотрим пример, где это не так.

\begin{remark}
	Вырожденное топологическое пространство при $ \Abs{T} \Gr 2 $ не является метрическим.
\end{remark}

\begin{definition}[топология Зар\'{и}сского]
	Пусть $ n \in \N $, $ \F $ --- некоторое поле, Определим замкнутые множества следующим образом:
	пусть $ A \ssq \F^n $; тогда множество $ A $ замкнуто $\iaoi$ множество $ A $ является решением некоторой конечной
	системы полиномиальных уравнений над $ \F $. Определим $ \W $ как множество всех открытых множеств. Тогда
	множество $ \W $ --- \emph{топология Зарисского} на $ \F^n $.
\end{definition}

\begin{restatable}[из домашнего задания]{exercise}{ExcIX}
	Докажите, что множество $ \W $ является топологией.
\end{restatable}

\begin{statement}
	Пусть $ \F = \R $ или $ \F = \C $. Тогда топология Зарисского на $ \F $ не может быть получена с помощью
	введения на $ \F $ метрики.
\end{statement}

\begin{proof}
	Решения систем полиномиальных уравнений --- либо пустое множество, либо конечное, либо вся числовая прямая
	(комплексная плоскость). По определению эти множества замкнуты, то есть дополнения до них открыты.
	Таким образом, открыты только пустое множество и все множества, дополнения до которых конечны.
	Пусть теперь $ U_1, U_2 \in \W $, $ U_1, U_2 \neq \Emp $. Предположим, что $ U_1 \cap U_2 = \Emp $. Ясно, что
	такого не может быть, (из $ U_1, U_2 \neq \Emp $ следует, что их дополнения конечны; взяв от обеих частей равенства
	$ U_1 \cap U_2 = \Emp $ дополнения, получаем, что конечное множество равно бесконечному).
	Таким образом, два непустых открытых множества всегда пересекаются. Отсюда следует, что на $ \F $ нельзя
	ввести метрику, так как в противном случае для двух разных точек можно было бы найти непересекающиеся окрестности.
\end{proof}

\subsection{Гомотопия и гомотопность отображений}

\begin{definition} Пусть $ n \in \N $, $ \mu \colon \R^{2 n} \to \R $ --- некоторая метрика, обозначим норму числа
	$ x \in \R^n $ через $ \Norm{x} = \mu(x, x) $, тогда:
	\begin{itemize}
		\item $ \B^n = \Set{ x \in \R^n \mid \Norm{x} < 1 } $ --- \emph{открытый шар};
		\item $ \D^n = \Set{ x \in \R^n \mid \Norm{x} \Le 1 } $ --- \emph{диск};
		\item $ \S^n = \Set{ x \in \R^{n+1} \mid \Norm{x} = 1 } = \D^{n+1} \wo \B^{n+1} $ --- \emph{сфера}.
	\end{itemize}
\end{definition}

\begin{remark}
	$ \B^0 = \D^0 $ --- одна точка, $ \S^0 $ --- пара точек, $ \S^1 $ --- окружность, $ \S^2 $ --- сфера.
\end{remark}

\begin{definition}
	\emph{Отображение $ f_k \colon \S^1 \to \S^1 $} --- это отображение, заданное следующим образом: для
	$ x = e^{2\pi i t} $ верно $ f_k(x) = e^{2 \pi i k t} $.
\end{definition}

Есть предположение, что если $ k_1 \neq k_2 $, то не существует однопараметрического семейства отображений (то есть
семейства отображений $ g_t $, где $ t \in [0; 1] $) окружности в окружность, которое бы начиналось в $ f_{k_1} $ и
заканчивалось в $ f_{k_2} $ (то есть $ g_0 = f_{k_1} $ и $ g_1 = f_{k_2} $ ). В дальнейшем мы формулизуем и докажем
это утверждение.

\end{document}